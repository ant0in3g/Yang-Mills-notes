\documentclass{beamer}

\usepackage[utf8]{inputenc}
\usepackage{default}
\usepackage{color}




%%%%%%%%%%%%%%%%%%%%%%%%%%%%%%%%%%%%%%%%%%%%%%%%%%%%%%%%%%%%%%%%%%%%%%%%%%%%%%%
%%%%%%%%%%%%%%%%%%%%%%%%%%%%      Theme     %%%%%%%%%%%%%%%%%%%%%%%%%%%%%%%%%%%
%%%%%%%%%%%%%%%%%%%%%%%%%%%%%%%%%%%%%%%%%%%%%%%%%%%%%%%%%%%%%%%%%%%%%%%%%%%%%%%
%\usepackage{movie15}
%\usepackage{beamerthemesplit}
%\usepackage{beamerthemeMalmoe}
%\usepackage{beamerinnerthemerounded}
\usepackage{beamerouterthememiniframes}
%\usepackage{beamerouterthemeshadow}
%\usepackage{beamerouterthemesidebar}
%\usepackage{beamerthemeAnnArbor}
%\usepackage{beamerthemeBerkeley}
%\usepackage{beamerthemeCambridgeUS}
%\usepackage{beamerthemeclassic}
%\usepackage{beamerthemeCopenhagen}
%\usepackage{beamerthemelined}
%\usepackage{beamerthemeMadrid}
%%%%%%%%%%%%%%%%%%%%%%%%%%%%%%%%%%%%%%%%%%%%%%%%%%%%%%%%%%%%%%%%%%%%%%%%%%%%%%%





\begin{document}

\title{\LARGE Renormalisation des théories de Jauge}  

\author{Antoine Géré \\ Responsable de stage : Jean-Christophe Wallet}

\date{26 juin 2012} 

\frame{\titlepage

} 

\frame{\frametitle{Table des matières} \tiny \tableofcontents} 

%%%%%%%%%%%%%%%%%%%%%%%%%%%%%%%%%%%%%%%%%%%%%%%%%%%%%%%%%%%%%%%%%%%%%%%%%%%%%%%%%%%%%%%%%%%%%%%%%%%%%%%%%%%%%%%%%%%%%%%%%%%%%%%%%%%%%%%%%%%%%%%%
\section{Classique} 
%%%%%%%%%%%%%%%%%%%%%%%%%%%%%%%%%%%%%%%%%%%%%%%%%%%%%%%%%%%%%%%%%%%%%%%%%%%%%%%%%%%%%%%%%%%%%%%%%%%%%%%%%%%%%%%%%%%%%%%%%%%%%%%%%%%%%%%%%%%%%%%%

%%%%%%%%%%%%%%%%%%%%%%%%%%%%%%%%%%%%%%%%%%%%%%%%%%%%%%%%%%%%%%%%%%%%%%%%%%%%%%%%%%%%%%%%%%%%%%%%%%%%%%%%%%%%%%%%%%%%%%%%%%%%%%%%%%%%%%%%%%%%%%%%
\subsection{Quelques considérations classiques}
%%%%%%%%%%%%%%%%%%%%%%%%%%%%%%%%%%%%%%%%%%%%%%%%%%%%%%%%%%%%%%%%%%%%%%%%%%%%%%%%%%%%%%%%%%%%%%%%%%%%%%%%%%%%%%%%%%%%%%%%%%%%%%%%%%%%%%%%%%%%%%%%

\frame{
\frametitle{Quelques considérations classiques} 
\scriptsize

Soit $G$ groupe de Lie compact de dimension n, et $\cal{G}$ son algèbre de Lie.\\
Lagrangien de Yang Mills,
\begin{eqnarray}
 && \mathcal{L}_{\small YM} = \frac{1}{8} tr\left(F_{\mu\nu}F_{\mu \nu}\right)  \\
 && \text{avec,} \hspace{2mm} F_{\mu \nu}=\partial_{\nu}A_{\mu} - \partial_{\mu}A_{\nu} + g[A_{\mu},A_{\nu}].
\end{eqnarray}
 On écrit $A_{\mu} = A_{\mu}^{a} T^{a}$.
\begin{eqnarray}
 F_{\mu \nu}^{a} = \partial_{\nu}A_{\mu}^{a} - \partial_{\mu}A_{\nu}^{a} + g f^{abd} A_{\mu}^{b} A_{\nu}^{d}
\end{eqnarray}
où $f^{abd}$ est la constante de structure de $\cal{G}$, completement antisymétrique.\\

\vspace{5mm}

La transformation de jauge défine comme suit :
\begin{eqnarray}
A_{\mu} \rightarrow  A^{\omega}_{\mu}(x) &=& \omega(x)A_{\mu}(x)\omega^{-1}(x) + ( \partial_{\mu} \omega(x) ) \omega^{-1}(x)
\end{eqnarray}
Transformation infinitésimale :
\begin{eqnarray}
 && \omega = 1+ u + \mathcal{O}(u^2)  \\
 && A^{\omega}_{\mu} = A_{\mu} - [A_{\mu},u] + \partial_{\mu}u  
\end{eqnarray}
$\mathcal{L}_{\small YM}$ est invariant sous cette transformation de jauge.
}

%%%%%%%%%%%%%%%%%%%%%%%%%%%%%%%%%%%%%%%%%%%%%%%%%%%%%%%%%%%%%%%%%%%%%%%%%%%%%%%%%%%%%%%%%%%%%%%%%%%%%%%%%%%%%%%%%%%%%%%%%%%%%%%%%%%%%%%%%%%%%%%%
\section{Quantification} 
%%%%%%%%%%%%%%%%%%%%%%%%%%%%%%%%%%%%%%%%%%%%%%%%%%%%%%%%%%%%%%%%%%%%%%%%%%%%%%%%%%%%%%%%%%%%%%%%%%%%%%%%%%%%%%%%%%%%%%%%%%%%%%%%%%%%%%%%%%%%%%%%

%%%%%%%%%%%%%%%%%%%%%%%%%%%%%%%%%%%%%%%%%%%%%%%%%%%%%%%%%%%%%%%%%%%%%%%%%%%%%%%%%%%%%%%%%%%%%%%%%%%%%%%%%%%%%%%%%%%%%%%%%%%%%%%%%%%%%%%%%%%%%%%%
\subsection{Quantification Lagrangienne}
%%%%%%%%%%%%%%%%%%%%%%%%%%%%%%%%%%%%%%%%%%%%%%%%%%%%%%%%%%%%%%%%%%%%%%%%%%%%%%%%%%%%%%%%%%%%%%%%%%%%%%%%%%%%%%%%%%%%%%%%%%%%%%%%%%%%%%%%%%%%%%%%

\frame{
\frametitle{Quantification Lagrangienne}
\scriptsize

Soit $\cal{I}$ l'action invariante de jauge du champ de Yang Mills,
\begin{eqnarray}
 \mathcal{I}_{YM} = \frac{1}{8} \int dx \hspace{1mm} tr ( F_{\mu \nu} F_{\mu \nu} ).
\end{eqnarray}
Matrice $\cal{S}$, opérateur d'évolution ''globale'' :
\begin{eqnarray}
 \mathcal{S} &=& \int exp\left( i \mathcal{I}_{YM} \right) dA_{\mu}  ,
\end{eqnarray}
Intégration sur toutes les configuratons des $A_{\mu}$, \\
or $exp\left( i \mathcal{I}_{YM} \right)$ prend la même valeur pour tout les $A_{\mu}$ qui se déduisent 
l'un de l'autre par une transformation de jauge, 
\begin{center}
$\mathcal{S}$ est donc infini !  
\end{center}
Fixage de jauge.\\
$\Rightarrow$ Méthode de Faddeev et Popov.
}

%%%%%%%%%%%%%%%%%%%%%%%%%%%%%%%%%%%%%%%%%%%%%%%%%%%%%%%%%%%%%%%%%%%%%%%%%%%%%%%%%%%%%%%%%%%%%%%%%%%%%%%%%%%%%%%%%%%%%%%%%%%%%%%%%%%%%%%%%%%%%%%%
\subsection{Méthode de Faddeev et Popov}
%%%%%%%%%%%%%%%%%%%%%%%%%%%%%%%%%%%%%%%%%%%%%%%%%%%%%%%%%%%%%%%%%%%%%%%%%%%%%%%%%%%%%%%%%%%%%%%%%%%%%%%%%%%%%%%%%%%%%%%%%%%%%%%%%%%%%%%%%%%%%%%%


\frame{
\frametitle{Méthode de Faddeev et Popov}
\scriptsize

Méthode de Faddeev et Popov, ou comment réécrire 1 !

\vspace{5mm}

Idée Principale : \hspace{1cm}  $ \int_{\mathbb{R}} \delta ( f(x) ) dx = \frac{1}{|f'(x)|}  $.  \\
\vspace{2mm}
On va exploiter cette relation !

\vspace{3mm}
On réécrie 1 ''dans la jauge de Lorentz''.
\begin{eqnarray}
 && \Delta^{-1}(A_{\mu}):= \int \delta \left(\partial_{\mu}A^{\omega}_{\mu}\right)d\omega  \\
 && \text{avec} \hspace{5mm} \Delta(A_{\mu}) = 
 \left|det\left(\frac {\partial_{\mu}A^{\omega}_{\mu}}{\partial_{\mu}\omega}\right)\right| = det(M)
\end{eqnarray}
On peut écrire : \\
\vspace{-8mm}
\begin{eqnarray}
 \mathcal{S} &=& \int exp\left( i \mathcal{I}(A_{\mu}) \right) dA_{\mu} \hspace{2mm} 
 \Delta(A_{\mu}) \delta \left(\partial_{\mu}A_{\mu}^{\omega}\right) d\omega \\
             &=& \int d\omega \int exp\left( i \mathcal{I}(A_{\mu}) \right) dA_{\mu} \hspace{2mm} 
 \Delta(A_{\mu}) \delta \left(\partial_{\mu}A_{\mu}\right)  \\
 \mathcal{S} &=& N^{-1} \int exp\left( i \mathcal{I}(A_{\mu}) \right) dA_{\mu} \hspace{2mm} 
 \Delta(A_{\mu}) \delta \left(\partial_{\mu}A_{\mu}\right)
\end{eqnarray}

}

\frame{
\scriptsize

$\bullet$ Jauge plus générale : $ \partial_{\mu}A_{\mu} - b(x) = 0 $, 
\begin{eqnarray}
\mathcal{S} &=& N^{-1} \int exp\left( i \mathcal{I}(A_{k}) \right) \Delta(A_{\mu}) \delta \left(\partial_{\mu}A_{\mu} - b(x) \right) dA_{\mu}
\end{eqnarray}
Moyenne sur $b(x)$ avec un poids gaussien $\int \frac{i}{2\alpha} b^2  dx$ : 
\begin{eqnarray*}
\mathcal{S} &=& N^{-1} \int exp\left( i \mathcal{I}(A_{k}) - \int \frac{i}{2\alpha} (\partial_{\mu}A_{\mu})^2  dx \right)
\Delta(A_{\mu}) \delta \left(\partial_{\mu}A_{\mu} \right) dA_{\mu}
\end{eqnarray*}
$\bullet$ On peut écrire $\Delta(A_{\mu})$ à l'aide des variables de grassmann de la façon suivante :
\begin{equation}
 \Delta(A_{\mu}) := det(M) = \int exp \left(i \int \overline{c}^{a}(x)M^{ab}c^{b}(x) dx \right) d\overline{c} dc 
\end{equation}
\vspace{1mm}
Finalement :
\begin{eqnarray}
 \mathcal{S} &=& \int exp\left(i \int \left[  
\frac{-1}{4} F^{a}_{\mu\nu}(x)F^{a}_{\mu \nu}(x) + \overline{c}^{a}(x)M^{ab}c^{b}(x) - \frac{1}{\alpha} (\partial_{\mu}A_{\mu})^2  
\right] dx \right) dA_{\mu} dc d\overline{c}  \nonumber  \\
\end{eqnarray}

}


%%%%%%%%%%%%%%%%%%%%%%%%%%%%%%%%%%%%%%%%%%%%%%%%%%%%%%%%%%%%%%%%%%%%%%%%%%%%%%%%%%%%%%%%%%%%%%%%%%%%%%%%%%%%%%%%%%%%%%%%%%%%%%%%%%%%%%%%%%%%%%%%
\subsection{Règles de Feynman}
%%%%%%%%%%%%%%%%%%%%%%%%%%%%%%%%%%%%%%%%%%%%%%%%%%%%%%%%%%%%%%%%%%%%%%%%%%%%%%%%%%%%%%%%%%%%%%%%%%%%%%%%%%%%%%%%%%%%%%%%%%%%%%%%%%%%%%%%%%%%%%%%

\frame{
\frametitle{Propagateurs}
\scriptsize



\vspace{1cm}



}

\frame{
\frametitle{Vertex}
\scriptsize



\vspace{5mm}



\vspace{5mm}



}

%%%%%%%%%%%%%%%%%%%%%%%%%%%%%%%%%%%%%%%%%%%%%%%%%%%%%%%%%%%%%%%%%%%%%%%%%%%%%%%%%%%%%%%%%%%%%%%%%%%%%%%%%%%%%%%%%%%%%%%%%%%%%%%%%%%%%%%%%%%%%%%%
\section{BRST}
%%%%%%%%%%%%%%%%%%%%%%%%%%%%%%%%%%%%%%%%%%%%%%%%%%%%%%%%%%%%%%%%%%%%%%%%%%%%%%%%%%%%%%%%%%%%%%%%%%%%%%%%%%%%%%%%%%%%%%%%%%%%%%%%%%%%%%%%%%%%%%%%

%%%%%%%%%%%%%%%%%%%%%%%%%%%%%%%%%%%%%%%%%%%%%%%%%%%%%%%%%%%%%%%%%%%%%%%%%%%%%%%%%%%%%%%%%%%%%%%%%%%%%%%%%%%%%%%%%%%%%%%%%%%%%%%%%%%%%%%%%%%%%%%%
\subsection{Symétrie BRST (Becchi, Rouet, Stora, et Tyupkin)}
%%%%%%%%%%%%%%%%%%%%%%%%%%%%%%%%%%%%%%%%%%%%%%%%%%%%%%%%%%%%%%%%%%%%%%%%%%%%%%%%%%%%%%%%%%%%%%%%%%%%%%%%%%%%%%%%%%%%%%%%%%%%%%%%%%%%%%%%%%%%%%%%

\frame{
\frametitle{Symétrie BRST (Becchi, Rouet, Stora, et Tyupkin) }
\scriptsize

Action fixée de jauge :
\begin{equation}
 \mathcal{L} = \mathcal{L}_{YM} +\mathcal{L}_{fix.} +\mathcal{L}_{FP}
\end{equation}
$\mathcal{I}$ n'est pas invariante de jauge.
\begin{eqnarray*}
 && \mathcal{L}_{YM} = \frac{1}{8} tr \left[ F_{\mu \nu}F_{\mu \nu} \right] = \frac{-1}{4} F^{a}_{\mu \nu}F^{a}_{\mu \nu}
\hspace{5mm} \rightarrow \hspace{5mm} \textbf{lagrangien de Yang Mills}\\
 && \mathcal{L}_{fix.} = - \frac{1}{4\alpha} (\partial_{\mu} A_{\mu})^{2} = \frac{1}{2\alpha} (\partial_{\mu} A^{a}_{\mu})^{2}
\hspace{5mm} \rightarrow \hspace{5mm} \textbf{lagrangien qui fixe de jauge}\\
 && \mathcal{L}_{FP} =  \overline{c}^{a} \left( \Box c^{a} -g f^{abd} \partial_{\mu} \left( A^{b}_{\mu} c^{d} \right) \right)
\hspace{5mm} \rightarrow \hspace{5mm} \textbf{lagrangien des fantômes}
\end{eqnarray*}
Seule $\mathcal{L}_{YM}$ est inavariant de jauge. \\
$\mathcal{L}_{fix.}$ et $\mathcal{L}_{FP}$ introduisent des termes suppémentaires !

}

\frame{
\scriptsize

$\bullet$ Modifier la transformation de jauge pour conserver une symétrie ! On pose : $ u = c$.
\begin{eqnarray}
  A^{\omega}_{\mu} &\simeq& A_{\mu} - [A_{\mu},c] + \partial_{\mu}c + \mathcal{O} (c) \\
                   &\simeq& A_{\mu}  + D_{\mu} c + \mathcal{O} (c)
\end{eqnarray}
$ \mathcal{L}_{YM} $ reste invariant. \\
Mais :
\vspace{-0.6cm}
\begin{eqnarray}
 \mathcal{L}_{fix.}  \rightarrow \frac{1}{2 \alpha} \left( \partial_{\mu} A^{a}_{\mu} \right)^{2} 
+ \frac{1}{\alpha} \left(  \partial_{\mu}D_{\mu} c^{a}  \right) \left( \partial_{\mu} A^{a}_{k} \right) + \mathcal{O} (c^{a})  
\end{eqnarray}



$\bullet$ Définir la transformations des fantômes, pour conpenser les termes supplémentaires ! 
\begin{eqnarray}
 \mathcal{L}_{FP} &=&  \overline{c}^{a} \left( \Box c^{a} -g f^{abd} \partial_{\mu} \left( A^{b}_{\mu} c^{d} \right) \right) \\
                  &=&  \overline{c}^{a} (\partial_{\mu}D_{\mu} c^{a} )
\end{eqnarray}
Si on pose que $\overline{c}$ se transforme comme :
\begin{eqnarray}
 \overline{c}^{a} &\rightarrow& \overline{c}^{a} - \frac{1}{\alpha} \left( \partial_{\mu} A^{a}_{\mu} \right) ,
\end{eqnarray}
compense exactement le terme supplémentair introduit par $\mathcal{L}_{fix.}$.

}

\frame{
\scriptsize
$\bullet$ Laisser invariant $\mathcal{L}_{FP}$. \\
C'est le cas si : 
\begin{eqnarray}
 c^{a} \rightarrow c^{a} - \frac{1}{2} f^{abd} c^{b} c^{d} 
\end{eqnarray}

\vspace{2mm}
$\bullet$ \underline{Transformations BRST}
\begin{eqnarray}
                A_{\mu}^{a}         \rightarrow       A_{\mu}^{a}           +  \delta A_{\mu}^{a}
\hspace{5mm}  &\text{avec}&   \hspace{5mm}  
\delta  A_{\mu}^{a} := s  A_{\mu}^{a} = D_{\mu} c^{a} \\
                \overline{c}^{a}                \rightarrow      \overline{c}^{a}                 +  \delta \overline{c}^{a} 
\hspace{5mm}  &\text{avec}&   \hspace{5mm}    
\delta  \overline{c}^{a} := s  \overline{c}^{a} = - \frac{1}{\alpha} ( \partial_{\mu} A_{\mu}^{a} ) \\
               c^{a}     \rightarrow      c^{a}      +  \delta  c^{a}   
\hspace{5mm}  &\text{avec}&   \hspace{5mm} 
\delta  c^{a} := s  c^{a} = - \frac{1}{2} f^{abd} c^{b} c^{d}  .
\end{eqnarray}
Le jacobien vaut 1.\\

$\Rightarrow$ La matrice $\cal{S}$ est invariante sous BRST ! 

}




%%%%%%%%%%%%%%%%%%%%%%%%%%%%%%%%%%%%%%%%%%%%%%%%%%%%%%%%%%%%%%%%%%%%%%%%%%%%%%%%%%%%%%%%%%%%%%%%%%%%%%%%%%%%%%%%%%%%%%%%%%%%%%%%%%%%%%%%%%%%%%%%
\section{Renormalisation}
%%%%%%%%%%%%%%%%%%%%%%%%%%%%%%%%%%%%%%%%%%%%%%%%%%%%%%%%%%%%%%%%%%%%%%%%%%%%%%%%%%%%%%%%%%%%%%%%%%%%%%%%%%%%%%%%%%%%%%%%%%%%%%%%%%%%%%%%%%%%%%%%

%%%%%%%%%%%%%%%%%%%%%%%%%%%%%%%%%%%%%%%%%%%%%%%%%%%%%%%%%%%%%%%%%%%%%%%%%%%%%%%%%%%%%%%%%%%%%%%%%%%%%%%%%%%%%%%%%%%%%%%%%%%%%%%%%%%%%%%%%%%%%%%%
\subsection{Calcul d'un diagramme}
%%%%%%%%%%%%%%%%%%%%%%%%%%%%%%%%%%%%%%%%%%%%%%%%%%%%%%%%%%%%%%%%%%%%%%%%%%%%%%%%%%%%%%%%%%%%%%%%%%%%%%%%%%%%%%%%%%%%%%%%%%%%%%%%%%%%%%%%%%%%%%%%

\frame{
\frametitle{Calcul d'un diagramme}
\scriptsize


En appliquant les règles de Feynman, on obtient :
\begin{eqnarray}
 \Pi^{ab}_{\mu\nu}(p) &=&  - i g^{2}  \delta^{ab} \int \frac{d^{4}k}{\left(2\pi\right)^4} 
[g_{\mu\nu}\left((p+k)^{2} +(k-2p)^2\right) \nonumber \\ && +  p_{\mu}p_{\nu}\left(n-6\right) +k_{\mu}k_{\nu}\left(4n-6\right)  \nonumber  \\
&&  +(3-2n)(p_{\nu}k_{\mu}+p_{\mu}k_{\nu})]\frac{1}{k^2+i0}\frac{1}{(k-p)^2+i0}
\end{eqnarray}

\vspace{3mm}

On s'aperçoit que $\Pi^{ab}_{\mu\nu}(p)$ diverge.

}

%%%%%%%%%%%%%%%%%%%%%%%%%%%%%%%%%%%%%%%%%%%%%%%%%%%%%%%%%%%%%%%%%%%%%%%%%%%%%%%%%%%%%%%%%%%%%%%%%%%%%%%%%%%%%%%%%%%%%%%%%%%%%%%%%%%%%%%%%%%%%%%%
\subsection{Comptage en Puissance}
%%%%%%%%%%%%%%%%%%%%%%%%%%%%%%%%%%%%%%%%%%%%%%%%%%%%%%%%%%%%%%%%%%%%%%%%%%%%%%%%%%%%%%%%%%%%%%%%%%%%%%%%%%%%%%%%%%%%%%%%%%%%%%%%%%%%%%%%%%%%%%%%

\frame{
\frametitle{Comptage en Puissance}
\scriptsize
\vspace{5mm}
L'amplitude type :
\begin{eqnarray}
 && J(k)= \int \prod_{ {1\le q\le n\atop} } \delta \left( \sum p - k_{q} \right) \prod_{1\le l\le L\atop} D_{l}(p_{l}) d_{p_{l}}  \\
 && \hspace{3mm} \text{où,} \hspace{3mm} D_{l}(p_{l}) = \frac{Z(p_{l})}{ m_{l}^{2}-p_{l}^{2}  }.
\end{eqnarray}
où $Z$ est un polynôme en $p_{l}$ de degré $r_{l}$.\\
\vspace{5mm}
On établie tout d'abord quelques notations :\\
\hspace{1cm} $q$ : nombre de vertex \\
\hspace{1cm} $n$ : nombre total de vertex \\
\hspace{1cm} $l$ : nombre de ligne interne \\
\hspace{1cm} $L$ : nombre total de ligne \\
\hspace{1cm} $m$ : nombre de dérivée à chaque vertex \\
\hspace{1cm} $d$ : dimension \\
}

\frame{
\scriptsize
Scaling :
\begin{eqnarray}
 p_i , k_i \rightarrow a p_i , a k_i
\end{eqnarray}
J  $\rightarrow$ $a^{\omega} J$, où $\omega$ est appelé l'index. \\
\vspace{3mm}
\hspace{1cm} $\omega > 0$, le comptage en puissance prévoit une divergence. \\
\hspace{1cm} $\omega = 0$, le comptage en puissance prévoit une divergence logaritmique. \\
\hspace{1cm} $\omega < 0$, c'est sûr ça converge ! \\

\begin{eqnarray}
 \omega &=& \sum_{{1\le l\le L\atop}} (r_{l}-2) + d(L-(n-1)) + n.m\\
 &=& \sum_{{1\le l\le L\atop}} (r_{l}-2+d) - d(n-1) + n.m
\end{eqnarray}

}

\frame{
\scriptsize

On est à présent  dans le cas de Yang-Mills, on considère alors :\\
\vspace{2mm}
\hspace{1cm} $L^{A}_{in}$ : nombre de ligne interne de $A$.\\
\hspace{1cm} $L^{c}_{in}$ : nombre de ligne interne de $c$.\\
\hspace{1cm} $L^{A}_{ex}$ : nombre de ligne externe de $A$.\\
\hspace{1cm} $L^{c}_{ex}$ : nombre de ligne externe de $c$.\\
\hspace{1cm} $n_{4}$ : nombre total de vertex $AAAA$.\\
\hspace{1cm} $m_{4}$ : nombre de dérivée à chaque vertex $AAAA$, ici $m_{4}=0$.\\
\hspace{1cm} $n_{3}$ : nombre total de vertex $AAA$.\\
\hspace{1cm} $m_{3}$ : nombre de dérivée à chaque vertex $AAA$, ici $m_{3}=1$.\\
\hspace{1cm} $n_{c}$ : nombre total de vertex $c$.\\
\hspace{1cm} $m_{c}$ : nombre de dérivée à chaque vertex $c$, ici $m_{c}=1$.\\
\hspace{1cm} $r_{l}=0$ : degré de $Z$.\\

de plus on a :
\begin{eqnarray}
 2L^{A}_{in} + L^{A}_{ex} = 4n_{4}+3n_{3}+n_{c}  \hspace{3mm} &et& \hspace{3mm}   2L^{c}_{in} + L^{c}_{ex} = 2n_{c}
\end{eqnarray}
et donc :
\begin{eqnarray}
 \omega &=& d -L^{A}_{ex}-L^{c}_{ex} - n_{4} (d-4) - n_{3} (d-4) - n_{c} (d-4)
\end{eqnarray}
Cas particulier $d=4$, $\omega$ depend uniquement du nombre de ligne externe. \\
On a donc la relation suivante :
\begin{eqnarray}
 \omega &=& 4-L^{A}_{ex}-L^{c}_{ex}
\end{eqnarray}

}

\frame{
\scriptsize

Les seules diagrammes divergents dans la théorie de Yang Mills sont donc les suivants :



}

%%%%%%%%%%%%%%%%%%%%%%%%%%%%%%%%%%%%%%%%%%%%%%%%%%%%%%%%%%%%%%%%%%%%%%%%%%%%%%%%%%%%%%%%%%%%%%%%%%%%%%%%%%%%%%%%%%%%%%%%%%%%%%%%%%%%%%%%%%%%%%%%
\subsection{Régularisation dimmensionnelle}
%%%%%%%%%%%%%%%%%%%%%%%%%%%%%%%%%%%%%%%%%%%%%%%%%%%%%%%%%%%%%%%%%%%%%%%%%%%%%%%%%%%%%%%%%%%%%%%%%%%%%%%%%%%%%%%%%%%%%%%%%%%%%%%%%%%%%%%%%%%%%%%%

\frame{
\frametitle{Définition}
\scriptsize

\begin{eqnarray}
 J(k) = \int d^{4}p \hspace{1mm} f(p,k)  &\rightarrow& J(k,d) = \int d^{d}p \hspace{1mm} f(p,k)
\end{eqnarray}
Pour définir la régularisation dimenssionnelle nous nous donnons ces trois conditions :
\begin{enumerate}
\item translation
          \begin{eqnarray}
           \int d^{d}p \hspace{1mm} F(p+q) = \int d^{d}p \hspace{1mm} F(p)
          \end{eqnarray}
\item expansion
          \begin{eqnarray}
           \int d^{d}p \hspace{1mm} F(a.p) = |a^{-d} | \int d^{d}p \hspace{1mm} F(p)
          \end{eqnarray}
\item factorisation
          \begin{eqnarray}
           \int d^{d_1}p d^{d_2}q \hspace{1mm} f(p) g(q) = \int d^{d_1}p \hspace{1mm} f(p)  \int d^{d_2}q \hspace{1mm} g(q)
          \end{eqnarray}
\end{enumerate}
}

\frame{
\frametitle{Régularisation dimenssionnelle d'un diagramme}
\scriptsize



Paramétrisation de Feynman :
\begin{eqnarray}
 \frac{1}{k^2+(p-k)^2} = \int_0^{1} dz \frac{1}{\left[k^2(1-z)+(p-k)^2z\right]^2}
\end{eqnarray} 
suivi du changement de variable : 
\begin{eqnarray}
 k \rightarrow k+pz
\end{eqnarray}
La fonction beta :  
\begin{eqnarray}
 B(x,y) &=& \int_{0}^{1} dz z^{x-1} (1-z)^{y-1} = \int_{0}^{+\infty}  dt t^{x-1} (1+t)^{-x-y} \\
            &=& \frac{\Gamma(x) \Gamma(y)}{\Gamma(x+y)}
\end{eqnarray}
}

\frame{
\scriptsize

\begin{eqnarray}
\Pi^{ab}_{\mu\nu}(p) = \frac{-i g^2 \delta^{ab}}{\left(4\pi\right)^{d/2}}   \Bigg[ \nonumber \\
&& \vspace{2mm} \hspace{-2cm} g_{\mu\nu}p^2  \bigg(
5\frac{\Gamma(\frac{d}{2}-1)\Gamma(\frac{d}{2}-1)}{\Gamma(d-2)} 
-2\frac{\Gamma(\frac{d}{2})\Gamma(\frac{d}{2}-1)}{\Gamma(d-1)}
+2\frac{\Gamma(\frac{d}{2}+1)\Gamma(\frac{d}{2}-1)}{\Gamma(d)}       \nonumber   \\ 
&& +\frac{6(d-1)}{2-d}\frac{\Gamma(\frac{d}{2})\Gamma(\frac{d}{2})}{\Gamma(d)} \bigg)  \nonumber \\
&& \vspace{2mm} \hspace{-2cm} - p_{\mu}p_{\nu}
\bigg((4n-6)\frac{\Gamma(\frac{n}{2})\Gamma(\frac{d}{2})}{\Gamma(d)}
-(n-6)\frac{\Gamma(\frac{d}{2}-1)\Gamma(\frac{d}{2}-1)}{\Gamma(d-2)}  \bigg) \nonumber  \\
&& \vspace{2mm} \Bigg]. \left(-\frac{p^2}{\mu}\right)^{\frac{d}{2}-2}\Gamma(2-\frac{d}{2})   
\end{eqnarray}
}

\frame{
\scriptsize

$\left(-\frac{p^2}{\mu}\right)^{\frac{n}{2}-2}\Gamma(2-\frac{n}{2}) $ diverge en $n=4$ !  \\
Developpement asymptotique de la fonction Gamma.  $\epsilon = \left(\frac{n}{2}-2\right)$
\begin{eqnarray}
\Pi^{ab}_{\mu\nu}(p) &=& \frac{-i g^2 \delta^{ab}}{16 \pi^{2}} \left[
\left( g_{\mu\nu}p^2 - p_{\mu}p_{\nu} \right) \left( \frac{19}{6} \epsilon^{-1} + C_1 \right)
- \frac{1}{2} p_{\mu} p_{\nu} ( C_2 + \frac{1}{\epsilon} ) \right.  \nonumber  \\ 
    &+& \left. \left( g_{\mu \nu} p^2 - p_{\mu} p_{\nu} \right) \frac{19}{6} ln(\frac{\mu^2}{-p^2})
-\frac{1}{2} p_{\mu} p_{\nu}  ln( \frac{\mu^2}{- p^2} ) \right]
\end{eqnarray} 
}

\frame{
\scriptsize

\underline{\textbf{Diagramme à une boucle en les fantômes}}


\begin{eqnarray}
\Pi^{ab}_{\mu\nu}(p) &=& \frac{-i g^2 \delta^{ab}}{16 \pi^{2}} \left[
\left( g_{\mu\nu}p^2 - p_{\mu}p_{\nu} \right) \left( \frac{1}{6} \epsilon^{-1} + C_3 \right)
+ \frac{1}{2} p_{\mu} p_{\nu} ( C_4 + \frac{1}{\epsilon} ) \right.\\ 
    &+& \left. \left( g_{\mu \nu} p^2 - p_{\mu} p_{\nu} \right) \frac{1}{6} ln(\frac{\mu^2}{-p^2})
+\frac{1}{2} p_{\mu} p_{\nu}  ln( \frac{\mu^2}{- p^2} ) \right]
\end{eqnarray}
\vspace{3mm}
\underline{\textbf{Tadpole}}\\


Pas de contribution

}
\frame{
\scriptsize

\textbf{Finalement,}
\begin{equation}
\Pi^{ab}_{\mu \nu}(p) = \frac{-i g^2 \delta^{ab}}{16 \pi^{2}} 
\left( g_{\mu\nu}p^2 - p_{\mu}p_{\nu} \right) 
\left[ \frac{10}{3} \epsilon^{-1} + C_5 + \frac{10}{3} ln(\frac{\mu^2}{-p^2})  \right]
\end{equation}
On a réussi à isoler la divergence.\\
Le contre terme correspondant est donc :
\begin{eqnarray}
&&  (z_{2} - 1) =\frac{5g^2}{24 \pi^2} \epsilon^{-1} \\
&& \delta \mathcal{L}_{A^{2}} = (z_{2} - 1) \frac{1}{2} tr [ ( \partial_{\nu} A_{\mu} - \partial_{\mu} A_{\nu} )^2 ]
\end{eqnarray}
Lagrangien régularisé :
\begin{eqnarray}
\mathcal{L}_{R} &=& \mathcal{L} + \delta \mathcal{L}_{A^2} +  \delta \mathcal{L}_{A^3} +  \delta \mathcal{L}_{A^4}  
+  \delta \mathcal{L}_{\overline{c}c}  + \delta \mathcal{L}_{\overline{c}Ac}  
\end{eqnarray}


}

%%%%%%%%%%%%%%%%%%%%%%%%%%%%%%%%%%%%%%%%%%%%%%%%%%%%%%%%%%%%%%%%%%%%%%%%%%%%%%%%%%%%%%%%%%%%%%%%%%%%%%%%%%%%%%%%%%%%%%%%%%%%%%%%%%%%%%%%%%%%%%%%
\subsection{Preuve de la renormalisabilité à tous les ordres}
%%%%%%%%%%%%%%%%%%%%%%%%%%%%%%%%%%%%%%%%%%%%%%%%%%%%%%%%%%%%%%%%%%%%%%%%%%%%%%%%%%%%%%%%%%%%%%%%%%%%%%%%%%%%%%%%%%%%%%%%%%%%%%%%%%%%%%%%%%%%%%%%

\frame{
\frametitle{Preuve de la renormalisabilité à tous les ordres}
\scriptsize

Équation de Zinn-Justin :
\begin{eqnarray}
 \int dx \left[ \frac{\delta \Gamma}{\delta A^{a}_{\mu}} \frac{\delta \Gamma }{i \delta k^{a}_{\mu}} 
+ \frac{\delta \Gamma}{\delta c^{a}}\frac{\delta \Gamma }{i \delta l^{a}} \right] = 0.
\end{eqnarray}
 Développement en série de $\Gamma$
\begin{eqnarray}
 && \Gamma = \Gamma^{0} + \Gamma^{1} + \Gamma^{2} + ...  \\
 && \text{avec} \hspace{2mm} \Gamma^{n} = \Gamma_{R}^{n} + \Gamma^{n}_{div}
\end{eqnarray}
 On pose,
\begin{eqnarray}
 \Gamma^1 * \Gamma^2 = \int dx \left( \frac{\delta \Gamma^{1}}{\delta A} \frac{\delta \Gamma^{2}}{\delta k} 
+ \frac{\delta \Gamma^{1}}{\delta c} \frac{\delta \Gamma^{2}}{\delta l} \right)
\end{eqnarray}
Réécriture de l'équation de Zin-Justin :
\begin{eqnarray}
 \sum_{p=0}^{n} \Gamma^{(p)} * \Gamma^{(n-p)} = 0
\end{eqnarray}

}

\frame{
\scriptsize

\textbf{n=0}
\begin{eqnarray}
 && \Gamma^{0} = \int dx \left[ \mathcal{L} + k^{\mu}_{a} s A^{a}_{\mu} - l^{a} s c^{a} \right] := I\\
 && I*I = 0
\end{eqnarray}


\textbf{n=1}
\begin{equation}
 \sum_{p=0}^{1} \Gamma^{(p)} * \Gamma^{(n-p)} = 0 = \Gamma^{0} * \Gamma^{1} + \Gamma^{1} * \Gamma^{0}
\end{equation}
\begin{eqnarray*}
 && I * \Gamma^{1}_{R} + \Gamma^{1}_{R} * I = 0 \hspace{2mm} \rightarrow \hspace{2mm} \text{Cette relation est bien vérifiée !} \\
 && I * \Gamma^{1}_{div} + \Gamma^{1}_{div} * I = 0 \hspace{2mm} \rightarrow \hspace{2mm} \text{On veut que cette relation soit 
vérifiée.}
\end{eqnarray*}
On a envie d'écrire :
\begin{eqnarray}
I_{1} = I - \Gamma^{1}_ {div} &\text{mais on doit avoir :}&  I_{1} * I_{1} = 0.
\end{eqnarray}
Or $I_{1}$ est le terme renormalisé à l'ordre 1, et on a déjà calculé les contres termes à cet ordre. On a donc :
\begin{eqnarray}
 I_{1} \left( A, c, \overline{c}, k, l \right) = I \left( z_{2}^{\frac{1}{2}} A , \tilde{z}_{2}^{\frac{1}{2}} c , 
 \tilde{z}_{2}^{\frac{1}{2}} \overline{c} , \tilde{z}_{2}^{\frac{1}{2}} k , z_{4} l \right),
\end{eqnarray}
avec :
\begin{eqnarray}
 I_{1} *  I_{1}  &=&  0
\end{eqnarray}

}


\frame{
\scriptsize

\textbf{Le cas $n$ quelconque :}  \\
\begin{eqnarray}
 &&\Gamma^{(0)} * \Gamma^{(n)} + \Gamma^{(n)} * \Gamma^{(0)} = -\sum_{p=1}^{n-1} \Gamma^{(p)} * \Gamma^{(n-p)}
\end{eqnarray}
$\sum_{p=1}^{n-1} \Gamma^{(p)} * \Gamma^{(n-p)}$ est fini par hypothèse de récurrence.\\
Et donc on veut :
\begin{eqnarray}
  && I * \Gamma^{(n)}_{div} + \Gamma^{(n)}_{div} * I = 0
\end{eqnarray}
On défint $\sigma$ tel que :
\begin{eqnarray}
 && \sigma \Gamma^{n}_{div} := I * \Gamma^{n}_{div} + \Gamma^{n}_{div} * I = 0  \\
 && \hspace{-3.8cm} \text{On pose} \hspace{5mm} \left(x_{i}\right) = \left(A,c\right) 
 \text{ \hspace{1mm} et \hspace{1mm} }  \left(\theta_{i}\right) = \left(k,l\right) \\
 && \sigma := \frac{\partial I}{\partial x_{i}} \frac{\partial}{\partial \theta_{i}} + 
\frac{\partial I}{\partial \theta_{i}} \frac{\partial}{\partial x_{i}} \\
& \hspace{-5cm} \text{On a :} \hspace{2mm} \sigma^{2} = 0
\end{eqnarray}

}

\frame{
\scriptsize

On se rappelle que l'on a :
\begin{eqnarray*}
 I &=&  \int \left[
\frac{-1}{4g} F^{a}_{\mu\nu}(x)F^{a}_{\mu \nu}(x) 
+ \frac{1}{\alpha}(\partial_{\mu}A_{\mu}^{a})^2 .
+ \overline{c}^{a}(x)M^{ab}c^{b}(x) 
+ k_{\mu}^{a} D_{\mu}c^{a} 
+ \frac{1}{2}f^{abd}l^ac^bc^d \right]
\end{eqnarray*} 

On peut alors écrire :
\begin{eqnarray}
 \Gamma^{n}_{div} &=& \int dx \left[  L(A) +  \left(  \overline{c}^{a} \partial_{\mu}  + k_{\mu}^{b} \right) \Delta_{mu} c^{b} 
+ \frac{\gamma}{2} f^{abd} l^{a}c^{b}c^{d} \right]
\end{eqnarray}
Par analyse dimenssionnelle on peut déjà en déduire :
\begin{eqnarray}
 && d(\Delta) = 1 \hspace{3mm} \text{et} \hspace{3mm}  g(\Delta) = 0 \\
 && \text{donc,} \hspace{3mm} \Delta_{\mu} = \alpha \partial_{\mu} +\beta f^{abd} A^{d}_{\mu}
\end{eqnarray}
on peut alors calculer :
\begin{eqnarray}
 \sigma \Gamma^{n}_{div} = 0 \Leftrightarrow  
D_{\mu} \frac{\partial L}{\partial A_{\mu}} + (\beta-\alpha) f^{abd} A^{b}_{\mu}  \frac{\partial \mathcal{L}}{\partial A_{\mu}} = 0
\end{eqnarray}
dont la solution est :
\begin{eqnarray}
 L &=& a \mathcal{L} + (\beta - \alpha) A \frac{\mathcal{L}}{A}
\end{eqnarray}

}

\frame{
\scriptsize

On a donc au final :
\begin{eqnarray}
 \Gamma^{n}_{div} &=& \left[ \int dx \left( ( \beta - \alpha + \frac{a}{2}) \left[A \frac{\delta}{\delta A} 
+ l \frac{\delta}{\delta l} \right] + \frac{\alpha}{2} \left[k \frac{\delta}{\delta k} + c \frac{\delta}{\delta c} 
+ \overline{c} \frac{\delta}{\delta  \overline{c} } \right]  \right) - \frac{\lambda}{2} g \frac{\delta}{\delta g}    \right] \mathcal{I} \nonumber
\end{eqnarray}

\noindent
Or par hypothèse de récurence on a : 
\begin{eqnarray}
 I_{n-1} &=& I \left( z_{2,n-1}^{\frac{1}{2}}A , \tilde{z}^{\frac{1}{2}}_{2,n-1}c, 
\tilde{z}^{\frac{1}{2}}_{2,n-1} \overline{c} , \tilde{z}^{\frac{1}{2}}_{2,n-1}k , z_{4,n-1} l, z_{g,n-1} g  \right),
\end{eqnarray}
ce qui nous permet d'écrire : ($I_{n} = I_{n-1} - \Gamma^{n}_{div}$)
\begin{eqnarray}
 && z_{2,n}^{\frac{1}{2}} = z_{2,n-1}^{\frac{1}{2}} - \left( \beta - \alpha + \frac{a}{2} \right)\\
 && \tilde{z}_{2,n}^{\frac{1}{2}} = \tilde{z}_{2,n-1}^{\frac{1}{2}} - \left( \frac{\alpha}{2}\right) \\
 && z_{g,n}^{\frac{1}{2}} = z_{g,n-1}^{\frac{1}{2}} - \left( \frac{\lambda}{2}\right)
\end{eqnarray}

On a donc montré qu'à chaque ordre on a un nombre finis de contre termes. Ce qui achève la démonstration de la renormalisabilité de la théorie
à tout les ordres.

}

%%%%%%%%%%%%%%%%%%%%%%%%%%%%%%%%%%%%%%%%%%%%%%%%%%%%%%%%%%%%%%%%%%%%%%%%%%%%%%%%%%%%%%%%%%%%%%%%%%%%%%%%%%%%%%%%%%%%%%%%%%%%%%%%%%%%%%%%%%%%%%%%
\section{Apparté Non-Commutatif}
%%%%%%%%%%%%%%%%%%%%%%%%%%%%%%%%%%%%%%%%%%%%%%%%%%%%%%%%%%%%%%%%%%%%%%%%%%%%%%%%%%%%%%%%%%%%%%%%%%%%%%%%%%%%%%%%%%%%%%%%%%%%%%%%%%%%%%%%%%%%%%%%

%%%%%%%%%%%%%%%%%%%%%%%%%%%%%%%%%%%%%%%%%%%%%%%%%%%%%%%%%%%%%%%%%%%%%%%%%%%%%%%%%%%%%%%%%%%%%%%%%%%%%%%%%%%%%%%%%%%%%%%%%%%%%%%%%%%%%%%%%%%%%%%%
\subsection{Espace de Moyal}
%%%%%%%%%%%%%%%%%%%%%%%%%%%%%%%%%%%%%%%%%%%%%%%%%%%%%%%%%%%%%%%%%%%%%%%%%%%%%%%%%%%%%%%%%%%%%%%%%%%%%%%%%%%%%%%%%%%%%%%%%%%%%%%%%%%%%%%%%%%%%%%%

\frame{
\frametitle{Espace de Moyal}
\scriptsize

L'espace de Moyal est une déformation de l'espace euclidien. \\
On définit le produit de Moyal de la façon suivante :
\begin{eqnarray}
 \forall a,b \in \mathcal{S} , \hspace{1cm}  (a \star b) (x)  = \frac{1}{(\pi \theta)^{D}}  \int  d^{D}y    d^{D}z   a(x+y)   b(x+z)   
e^{-i y \tilde{z} }
\end{eqnarray}
où,
\begin{eqnarray}
 &\bullet&  \cal{S} (\mathbb{R}^{D}) \hspace{2mm} \text{est l'espace des fonctions Schwartz de} \hspace{2mm} \mathbb{R}^{D} \\
 &\bullet&  \tilde{z}_{\nu}  =  2 \Theta^{-1}_{\mu \nu} z_{\nu}  \\
 &\bullet&  y \tilde{z} = y_{\mu}  \tilde{z}_{\nu}  \\
 &\bullet&  \Theta_{\mu \nu}  =  \theta  \text{diag ( J,..., J) , une matrice} \hspace{2mm} D \times D  \\
 &\bullet&  J = 
\left(
  \begin{array}{ c c }
     0 & -1 \\
     1 & 0
  \end{array} 
\right)
\end{eqnarray}

}

%%%%%%%%%%%%%%%%%%%%%%%%%%%%%%%%%%%%%%%%%%%%%%%%%%%%%%%%%%%%%%%%%%%%%%%%%%%%%%%%%%%%%%%%%%%%%%%%%%%%%%%%%%%%%%%%%%%%%%%%%%%%%%%%%%%%%%%%%%%%%%%%
\subsection{Théorie de Yang Mills Non Commutative sur le plan de Moyal}
%%%%%%%%%%%%%%%%%%%%%%%%%%%%%%%%%%%%%%%%%%%%%%%%%%%%%%%%%%%%%%%%%%%%%%%%%%%%%%%%%%%%%%%%%%%%%%%%%%%%%%%%%%%%%%%%%%%%%%%%%%%%%%%%%%%%%%%%%%%%%%%%

\frame{
\frametitle{Theorie de Yang Mills sur le plan de Moyal}
\scriptsize

Action sur le plan de Moyal
\begin{eqnarray}
 \cal{I}  &=&  \int d^{2 } x  \left(  \frac{1}{4}  F_{\mu  \nu}  \star  F_{\mu  \nu}  + s \left( \overline{c} \star \partial_{\mu} A_{\mu}   
+  \frac{\alpha}{2} \overline{c} \star b \right)  \right)
\end{eqnarray}\\
\vspace{-3mm}
\textbf{Règles de feynman}\\
On considere le groupe SU(N), et le groupe U(1). Les indices sont noté par des lettres majuscule  
$A=(0,a)$, où $a$ designe la partie SU(N), 0 la partie U(1). \\

\underline{Propagateur:}
\begin{eqnarray}
 G^{ A^{A} B^{B} } (k)  &=&  \frac{ \delta^{AB} }{ k^2 } \left(  \delta_{\mu \nu}  - (1-\alpha)  \frac{ k_{\mu} k_{\nu} }{ k^2 } \right) \\
 G^{ \overline{c}^{A} c^{B} } (k)  &=& - \frac{ \delta^{AB} }{ k^2 }
\end{eqnarray}
\vspace{2mm}
\underline{Vertex} (2 fantomes et un champ de jauge)
\begin{eqnarray}
 V_{\mu}^{ \overline{c}^{0} A^{A} c^{B} } (q_1 , k_2 , q_3 ) &=& -2ig (2 \pi)^2 \delta^{2} ( q_1 +k_2 +q_3 ) q_{3_{\mu}} \frac{d^{AB0}}{2} 
sin( \frac{\epsilon}{2}  q_1 \tilde{q}_3 ) \\
 V_{\mu}^{ \overline{c}^{a} A^{b} c^{c} } (q_1 , k_2 , q_3 ) &=& -2ig (2 \pi)^2 \delta^{2} ( q_1 +k_2 +q_3 ) q_{3_{\mu}} \mathcal{F}^{acb}(q_1 , q_3)
\end{eqnarray}
}

\frame{
\scriptsize

Diagramme 1 boucle avec deux lignes externe en fantômes
\begin{eqnarray*}
 \omega_{00}(p) &=&  \int  \frac{d^D k}{(2 \pi)^D}
V_{\nu}^{ \overline{c}^{0} A^{B} c^{0} } (p-k , k , -p )
V_{\mu}^{ \overline{c}^{0} A^{A} c^{0} } (p,-k,k-p )
G^{ \overline{c}^{0} c^{0} } (k-p)
G^{ A^{A} A^{B} }_{\mu \nu } (k) 
\end{eqnarray*}
Divergence IR :
\begin{eqnarray}
 \lim\limits_{
\begin{array}{l}
D \to 2\\
p \to 0
\end{array}}   \omega_{00}(p)  &=&   - \frac{1}{2}  g^2  (2 \pi)^3  N^{-1} \delta^{00}
\int_{0}^{1}  dx  x
\left(
 \frac{1}{x(1-x)} 
-  \frac{1}{x(1-x)} 
\right)      
\end{eqnarray}
On a donc :
\begin{eqnarray}
 \lim\limits_{
\begin{array}{l}
D \to 2\\
p \to 0
\end{array}}   \omega_{00}(p)  &=&  0
\end{eqnarray}
Dans ce cas les divergences infrarouge ''s'auto-annulent.''









}

%%%%%%%%%%%%%%%%%%%%%%%%%%%%%%%%%%%%%%%%%%%%%%%%%%%%%%%%%%%%%%%%%%%%%%%%%%%%%%%%%%%%%%%%%%%%%%%%%%%%%%%%%%%%%%%%%%%%%%%%%%%%%%%%%%%%%%%%%%%%%%%%
\section{Conclusion}
%%%%%%%%%%%%%%%%%%%%%%%%%%%%%%%%%%%%%%%%%%%%%%%%%%%%%%%%%%%%%%%%%%%%%%%%%%%%%%%%%%%%%%%%%%%%%%%%%%%%%%%%%%%%%%%%%%%%%%%%%%%%%%%%%%%%%%%%%%%%%%%%

\frame{
\frametitle{Conclusion}

\textbf{Bilan}\\
\vspace{2mm}
$\bullet$ \hspace{1mm}Une opportunité d'approfondir et de completer mes connaissances en théories des champs.\\
\vspace{2mm}
$\bullet$ \hspace{1mm}Une introduction aux théories des champs non commutatives.\\
\vspace{1cm}
\textbf{Perspectives}\\
\vspace{2mm}
$\bullet$ \hspace{1mm}S'intéresser, dans le cas commtatif, aux brisures spontanées de symétries.\\
\vspace{2mm}
$\bullet$ \hspace{1mm}Dans un premier temps finir les caluls pour le cas D=2, et ensuite tenter de comprendre pleinement les théories 
de jauge sur ces espaces non commutatifs.

}


\frame{
\LARGE
\begin{center}
 Merci pour votre attention.
\end{center}
}

%%%%%%%%%%%%%%%%%%%%%%%%%%%%%%%%%%%%%%%%%%%%%%%%%%%%%%%%%%%%%%%%%%%%%%%%%%%%%%%%%%%%%%%%%%%%%%%%%%%%%%%%%%%%%%%%%%%%%%%%%%%%%%%%%%%%%%%%%%%%%%%%
%%%%%%%%%%%%         Annexes          %%%%%%%%%%%%%%%%%%%%%%%%%%%%%%%%%%%%%%%%%%%%%%%%%%%%%%%%%%%%%%%%%%%%%%%%%%%%%%%%%%%%%%%%%%%%%%%%%%%%%%%%%%
%%%%%%%%%%%%%%%%%%%%%%%%%%%%%%%%%%%%%%%%%%%%%%%%%%%%%%%%%%%%%%%%%%%%%%%%%%%%%%%%%%%%%%%%%%%%%%%%%%%%%%%%%%%%%%%%%%%%%%%%%%%%%%%%%%%%%%%%%%%%%%%%

\frame{

\frametitle{Annexe}

\textbf{Interprétation Géométrique}
\scriptsize
\vspace{2mm}\\
d : opérateur de De Rham ($d^2 = 0$), \hspace{1cm} $d : \Omega^{p,g} \rightarrow \Omega^{p+1,g} $\\
s : opérateur BRST ($s^2 = 0$), \hspace{1.9cm} $s : \Omega^{p,g} \rightarrow \Omega^{p,g+1}$ \\
\begin{eqnarray}
                A_{\mu}^{a}         \rightarrow       A_{\mu}^{a}           +  \delta A_{\mu}^{a}
\hspace{5mm}  &\text{avec}&   \hspace{5mm}  
\delta  A_{\mu}^{a} := s  A_{\mu}^{a} = D_{\mu} c^{a} \\
                c^{a}                \rightarrow      c^{a}                 +  \delta c^{a}   
\hspace{5mm}  &\text{avec}&   \hspace{5mm}    
\delta  c^{a} := s  c^{a} = - \frac{1}{\alpha} ( \partial_{\mu} A_{\mu}^{a} ) \\
                \overline{c}^{a}     \rightarrow      \overline{c}^{a}      +  \delta  \overline{c}^{a}   
                \hspace{5mm}  &\text{avec}&   \hspace{5mm} 
\delta  \overline{c}^{a} := s  \overline{c}^{a} = - \frac{1}{2} f^{abd} c^{b} c^{d}  .
\end{eqnarray}

\begin{figure}
\begin{center}
\begin{tabular}{|c|c|c|}
\hline
   Champs & Degré des formes & Degré en fantôme \\
   & p & q \\
\hline
\hline
 $A$ & +1 & 0 \\
\hline
 $c$ & 0 & +1 \\
\hline
 $\overline{c}$ & 0 & -1 \\
\hline
\hline
 $sA$ & +1 & \textcolor{red}{+1} \\
\hline
 $sc$ & 0 & +1 \textcolor{red}{+1} = 2 \\
\hline
\end{tabular}
\end{center}
\end{figure}

}

\frame{
\scriptsize

\begin{figure}
\begin{eqnarray}
 \tilde{A}  &:=& A^{1,0} + c^{0,1}\\
 \tilde{F}  &:=& \tilde{d} \tilde{A}^{1,1} + \frac{1}{2} \left[ \tilde{A}^{1,1} , \tilde{A}^{1,1} \right] := F
\end{eqnarray}
Deux des transformations BRST :
\begin{eqnarray}
  && sc = - \frac{1}{2} \left[ c , c \right] \\
  && sA = - \left( dc + [A,c] \right) = Dc ,
\end{eqnarray}
\end{figure}
}

\end{document}